\documentclass{nime-alternate} % Uncomment when publishing final version
% Uncomment when publishing final version
% \documentclass{nime-alternate}
% \usepackage{anonymize}
\usepackage[blind]{anonymize}
\usepackage[utf8]{inputenc}
\usepackage{hyperref}

\usepackage{color}
\usepackage{fancyvrb}
\newcommand{\VerbBar}{|}
\newcommand{\VERB}{\Verb[commandchars=\\\{\}]}
\DefineVerbatimEnvironment{Highlighting}{Verbatim}{commandchars=\\\{\}}
\usepackage{framed}
\definecolor{shadecolor}{RGB}{240,240,240}
\newenvironment{Shaded}{\begin{snugshade}}{\end{snugshade}}
\newcommand{\AlertTok}[1]{\textcolor[rgb]{0.94,0.16,0.16}{#1}}
\newcommand{\AnnotationTok}[1]{\textcolor[rgb]{0.56,0.35,0.01}{\textbf{\textit{#1}}}}
\newcommand{\AttributeTok}[1]{\textcolor[rgb]{0.77,0.63,0.00}{#1}}
\newcommand{\BaseNTok}[1]{\textcolor[rgb]{0.00,0.00,0.81}{#1}}
\newcommand{\BuiltInTok}[1]{#1}
\newcommand{\CharTok}[1]{\textcolor[rgb]{0.31,0.60,0.02}{#1}}
\newcommand{\CommentTok}[1]{\textcolor[rgb]{0.56,0.35,0.01}{\textit{#1}}}
\newcommand{\CommentVarTok}[1]{\textcolor[rgb]{0.56,0.35,0.01}{\textbf{\textit{#1}}}}
\newcommand{\ConstantTok}[1]{\textcolor[rgb]{0.00,0.00,0.00}{#1}}
\newcommand{\ControlFlowTok}[1]{\textcolor[rgb]{0.13,0.29,0.53}{\textbf{#1}}}
\newcommand{\DataTypeTok}[1]{\textcolor[rgb]{0.13,0.29,0.53}{#1}}
\newcommand{\DecValTok}[1]{\textcolor[rgb]{0.00,0.00,0.81}{#1}}
\newcommand{\DocumentationTok}[1]{\textcolor[rgb]{0.56,0.35,0.01}{\textbf{\textit{#1}}}}
\newcommand{\ErrorTok}[1]{\textcolor[rgb]{0.64,0.00,0.00}{\textbf{#1}}}
\newcommand{\ExtensionTok}[1]{#1}
\newcommand{\FloatTok}[1]{\textcolor[rgb]{0.00,0.00,0.81}{#1}}
\newcommand{\FunctionTok}[1]{\textcolor[rgb]{0.00,0.00,0.00}{#1}}
\newcommand{\ImportTok}[1]{#1}
\newcommand{\InformationTok}[1]{\textcolor[rgb]{0.56,0.35,0.01}{\textbf{\textit{#1}}}}
\newcommand{\KeywordTok}[1]{\textcolor[rgb]{0.13,0.29,0.53}{\textbf{#1}}}
\newcommand{\NormalTok}[1]{#1}
\newcommand{\OperatorTok}[1]{\textcolor[rgb]{0.81,0.36,0.00}{\textbf{#1}}}
\newcommand{\OtherTok}[1]{\textcolor[rgb]{0.56,0.35,0.01}{#1}}
\newcommand{\PreprocessorTok}[1]{\textcolor[rgb]{0.56,0.35,0.01}{\textit{#1}}}
\newcommand{\RegionMarkerTok}[1]{#1}
\newcommand{\SpecialCharTok}[1]{\textcolor[rgb]{0.00,0.00,0.00}{#1}}
\newcommand{\SpecialStringTok}[1]{\textcolor[rgb]{0.31,0.60,0.02}{#1}}
\newcommand{\StringTok}[1]{\textcolor[rgb]{0.31,0.60,0.02}{#1}}
\newcommand{\VariableTok}[1]{\textcolor[rgb]{0.00,0.00,0.00}{#1}}
\newcommand{\VerbatimStringTok}[1]{\textcolor[rgb]{0.31,0.60,0.02}{#1}}
\newcommand{\WarningTok}[1]{\textcolor[rgb]{0.56,0.35,0.01}{\textbf{\textit{#1}}}}

\newlength{\cslhangindent}
\setlength{\cslhangindent}{1.5em}
\newenvironment{cslreferences}%
  {\setlength{\parindent}{0pt}%
  \everypar{\setlength{\hangindent}{\cslhangindent}}\ignorespaces}%
  {\par}

\begin{document}

\conferenceinfo{NIME'20,}{July 21-25, 2020, Royal Birmingham Conservatoire, ~~~~~~~~~~~~ Birmingham City University, Birmingham, United Kingdom.}
\title{Algorithmic Pattern}

\numberofauthors{1}
%
\author{
\alignauthor
\anonymize{Alex McLean}\\
       \affaddr{\anonymize{Research Institute for the History of Science and Technology}}\\
       \affaddr{\anonymize{Deutches Museum}}\\
       \affaddr{\anonymize{Munich}}\\
       \email{\anonymize{alex@slab.org}}
}

\date{31st January 2020}

\maketitle
\begin{abstract}
This paper brings together two main perspectives on algorithmic
pattern. First, the writing of musical patterns in live coding
performance, and second, the weaving of patterns in textiles. In both
cases, algorithmic pattern is an interface between the human and the
outcome, where small changes have far-reaching impact on the results.

By bringing contemporary live coding and ancient textile approaches
together, we reach a common view of pattern as algorithmic movement
(e.g. looping, shifting, reflecting, interfering) in the making of
things. This works beyond the usual definition of pattern used in
musical interfaces, of mere repeating sequences. We conclude by
considering the place of algorithmic pattern in a wider activity of
making.

\end{abstract}

\keywords{Pattern, TidalCycles, Algorithmic Music, Textiles, Live Coding, Algorave}

%% \begin{CCSXML}
%% <ccs2012>
%%    <concept>
%%        <concept_id>10010405.10010469.10010475</concept_id>
%%        <concept_desc>Applied computing~Sound and music computing</concept_desc>
%%        <concept_significance>500</concept_significance>
%%        </concept>
%%    <concept>
%%        <concept_id>10010405.10010469.10010471</concept_id>
%%        <concept_desc>Applied computing~Performing arts</concept_desc>
%%        <concept_significance>500</concept_significance>
%%        </concept>
%%    <concept>
%%        <concept_id>10010405.10010469.10010474</concept_id>
%%        <concept_desc>Applied computing~Media arts</concept_desc>
%%        <concept_significance>500</concept_significance>
%%        </concept>
%%    <concept>
%%        <concept_id>10011007.10011006.10011008.10011009.10011012</concept_id>
%%        <concept_desc>Software and its engineering~Functional languages</concept_desc>
%%        <concept_significance>300</concept_significance>
%%        </concept>
%%  </ccs2012>
%% \end{CCSXML}

\ccsdesc[500]{Applied computing~Sound and music computing}
\ccsdesc[500]{Applied computing~Performing arts}
\ccsdesc[500]{Applied computing~Media arts}
\ccsdesc[300]{Software and its engineering~Functional languages}

% this line creates the CCS Concepts section.
\printccsdesc

$body$


$if(has-frontmatter)$
\backmatter
$endif$
$if(natbib)$
$if(bibliography)$
$if(biblio-title)$
$if(has-chapters)$
\renewcommand\bibname{$biblio-title$}
$else$
\renewcommand\refname{$biblio-title$}
$endif$
$endif$
$if(beamer)$
\begin{frame}[allowframebreaks]{$biblio-title$}
  \bibliographytrue
$endif$
  \bibliography{$for(bibliography)$$bibliography$$sep$,$endfor$}
$if(beamer)$
\end{frame}
$endif$

$endif$
$endif$
$if(biblatex)$
$if(beamer)$
\begin{frame}[allowframebreaks]{$biblio-title$}
  \bibliographytrue
  \printbibliography[heading=none]
\end{frame}
$else$
\printbibliography$if(biblio-title)$[title=$biblio-title$]$endif$
$endif$

$endif$
$for(include-after)$
$include-after$

$endfor$

\end{document}
