% NIME 2020 Music Proceedings Template

% Modified December 2019 by Joe Wright
% Created August 2019 by Niccolo Granieri
% This is file `music-proceedings-template.tex',
%
% The original source files were:
%
% samples.dtx  (with options: `acmsmall')
% 
% IMPORTANT NOTICE:
% 
% For the copyright see the source file.
% 
% Any modified versions of this file must be renamed
% with new filenames distinct from sample-acmsmall.tex.
% 
% For distribution of the original source see the terms
% for copying and modification in the file samples.dtx.
% 
% This generated file may be distributed as long as the
% original source files, as listed above, are part of the
% same distribution. (The sources need not necessarily be
% in the same archive or directory.)


% The first command in your LaTeX source must be the 
% \documentclass command.
\documentclass{nimemusic}


\usepackage{lipsum} %used to generate default text
% Rights management information.  This information is sent to you
% when you complete the rights form.  These commands have SAMPLE
% values in them; it is your responsibility as an author to replace
% the commands and values with those provided to you when you
% complete the rights form.
\setcopyright{cc4}
\nimeYear{2020}
\nimeMonth{7}
\nimeDOI{10.1145/1122445.1122456}
\whichNIME{NIME’20, July 21-25, 2020, Birmingham, United Kingdom}

%\fancypagestyle{firstPage}{%
%  \fancyhf{}
%  \renewcommand\headrulewidth{0pt}
%  \fancyfoot[R]{Some special text}
%}


% end of the preamble, start of the body of the document source.
\begin{document}

% The "title" command has an optional parameter,
% allowing the author to define a "short title" to be used in page headers.
\title{Title: Feedforward}



% The "author" command and its associated commands are used to define
% the authors and their affiliations.
% Of note is the shared affiliation of the first two authors, and the
% "authornote" and "authornotemark" commands
% used to denote shared contribution to the research.
\author{Name removed for peer review}
\affiliation{%
  \institution{--}
  \city{--}
  \country{--}
}
%\author{Alex McLean}
%\affiliation{%
%  \institution{Research Institute for the History of Science and Technology, Deutsches Museum}
%  \city{Munich}
%  \country{Germany}
%}

% By default, the full list of authors will be used in the page
% headers. Often, this list is too long, and will overlap
% other information printed in the page headers. This command allows
% the author to define a more concise list
% of authors' names for this purpose.
\renewcommand{\shortauthors}{Removed}


% Keywords. The author(s) should pick words that accurately describe
% the work being presented. Separate the keywords with commas.
\keywords{Live Coding, Feedback, Text Editor, TUI, TidalCycles}


% This command processes the author and affiliation and title
% information and builds the first part of the formatted document.
\maketitle
%\thispagestyle{firstPage}

\section{Program Notes}

This is an improvised, from-scratch live coding performance. The NIME
interface which this performance showcases is the new Feedfoward
editor for the TidalCycles live coding environment. Feedforward is
written in Haskell using the ncurses library for terminal-based user
interfaces. It runs on low-powered hardware including the Raspberry Pi
Zero, with formative testing of prototypes conducted with several
groups of children between the ages of 8 and 14.

Feedforward has a number of features designed to support improvised,
multi-pattern live coding. Individual Tidal patterns are addressable
with hotkeys for fast mute and unmuting. Each pattern has a stereo VU
meter, to aid the quick matching of sound to pattern within a mix.

In addition, TidalCycles has been extended to store context with each
event, so that source code positions in its polyrhythmic sequence
mini-notation are tracked. This allows steps to be highlighted in the
source code whenever they are active. This works even when Tidal
combinators have been applied to manipulate the timeline. Formal
evaluation has yet to take place, but this feature appears to support
learning of how pattern manipulations work in Tidal.

Feedforward and TidalCycles is free/open source software under a GPL
licence version 3.0.

\section{PERFORMANCE NOTES}

Space: The performance would work best as an algorave-style club
performance, but could also work in a concert setting.

Technical requirements: Projection to rear of stage, with HDMI cable
to laptop on stage. Full-range, stereo PA, with on stage stereo
monitoring. Venue provides stereo pair of XLR cables from my Radial
USB DI.

Performers: Solo laptop musician

Feasibility: I have improvised hundreds of from-scratch live coded
performances over the past 17 years. I have performed with this new
editor feedforward several times including a live streamed performance
watched by tens of thousands of people.

\section{Media Links}

A short demonstration video: \url{https://vimeo.com/388656525}

\begin{acks}
Removed for peer review.
\end{acks}

% The next two lines define the bibliography style to be used, and
% the bibliography file.
\end{document}
\endinput
%
% End of file `sample-acmsmall.tex'.
